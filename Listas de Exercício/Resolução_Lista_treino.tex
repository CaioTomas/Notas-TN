\documentclass[a4paper,12pt]{article}
\usepackage[utf8]{inputenc}
\usepackage{graphicx}
\usepackage{amsmath}
\usepackage{amssymb}
\usepackage{amsfonts}
\usepackage{textcomp}
\usepackage{amsthm}
\usepackage{subcaption}
\usepackage{booktabs}
\usepackage{float}
\usepackage{mathrsfs}
\usepackage{mathtools}
\usepackage{xfrac}
\usepackage{hyperref}
\usepackage{xfrac}
\usepackage{enumerate}
\usepackage{cancel}
\DeclareMathOperator{\tr}{tr}
\DeclareMathOperator{\aut}{Aut}
\DeclareMathOperator{\inn}{Inn}
\DeclareMathOperator{\stab}{stab}
\DeclareMathOperator{\orb}{orb}
\DeclareMathOperator{\Mod}{Mod}
\DeclareMathOperator{\Ker}{Ker}
\DeclareMathOperator{\mdc}{mdc}
\DeclareMathOperator{\mmc}{mmc}
\DeclareMathOperator{\syl}{Syl}
\DeclareMathOperator{\tor}{Tor}
\DeclareMathOperator{\Sim}{Sim}

% Times for rm and math | Helvetica for ss | Courier for tt
%\usepackage{mathptmx} % rm & math
%\usepackage[scaled=0.90]{helvet} % ss
%\usepackage{courier} % tt
%\normalfont
%\usepackage[T1]{fontenc}

\usepackage[all]{xy}

%\input xy
%\xyoption{all}

%\documentclass{standalone}

\usepackage{anonchap}
\usepackage[symbol]{footmisc}
\usepackage{anonchap}
\usepackage[Sonny]{fncychap}
%\usepackage[brazilian]{babel}
%\usepackage[portuguese]{babel}
\usepackage{geometry}
\geometry{a4paper, left=3cm, top=3cm, right=2cm, bottom=2cm}
\usepackage{multicol}
\usepackage{fancyhdr}
%\usepackage[center]{caption}


\newtheorem{theorem}{Teorema}[section]
\newtheorem*{definition}{Definição}
\newtheorem{corollary}{Corolário}[theorem]
\newtheorem{lemma}[theorem]{Lema}
\newtheorem{remark}{Observação}[section]
\newtheorem{deff}{Definição}[section]
\newtheorem{fact}{Fato}[section]
\newtheorem{exercise}{Exercício}%[section]
\newtheorem{example}{Exemplo}[section]
\newtheorem{prop}{Proposição}[section]
\newtheorem*{solution}{Solução}
\title{Resolução Lista de Exercícios para treinar}
\date{20/08/2020}
\author{Caio Tomás}
\begin{document}
	\maketitle 
	\begin{exercise}
		Mostre que se $a|b$ e $b|a$, então $a = \pm b$.
	\end{exercise}
	\begin{proof}
		Se $a|b$, então $b = k_1a$, $k_1\in\mathbb{Z}$. Como $b|a$, segue que $a$ é um múltiplo de $b$. Substituindo em $b = k_1a$, temos que $b = k_1k_2b$, o que implica $k_1k_2 = 1$. Como $k_1, k_2\in\mathbb{Z}$, então devemos ter $k_1 = 1 = k_2$ ou $k_1 = -1 = k_2$. Consequentemente, $a = \pm b$.
	\end{proof}
	\begin{exercise}
		Mostre que se $a|c$ e $b|d$ então $ab|cd$.
	\end{exercise}
	\begin{proof}
		Como $a|c$ e $b|d$, segue que $c = k_1a$ e $d = k_2b$, $k_1, k_2$ inteiros. Fazendo o produto de $c$ e $d$, temos $cd = (k_1k_2)(ab)$, ou seja, $ab|cd$.
	\end{proof}
	\begin{exercise}
		Sejam $a,b,c\in\mathbb{Z}$. Mostre que $\mdc(a,b) = \mdc(a+cb, b) $.
	\end{exercise}
	\begin{proof}
		Sejam $d_1 = \mdc(a,b)$ e $d_2 = \mdc(a+cb, b)$. Por definição, sabemos que $d_1|a$ e $d_1|b$. Logo, $d_1$ divide qualquer combinação inteira de $a$ e $b$, em particular $d_1|a + cb$. Portanto, pela definição do máximo divisor comum, $d_1\leq d_2$.
		\par Por outro lado, sabemos que $d_2| a+cb$ e $d_2|b$. Dessas relações, sabemos que
		\begin{align*} 
		\begin{cases}
		a+cb = k_1d_2 \\
		b = k_2d_2
		\end{cases} \Rightarrow 
		a = (k_1 - ck_2)d_2 \Leftrightarrow d_2|a
		\end{align*} 
		\par Portanto, $d_2\leq d_1$ e segue que $d_1 = d_2$. Analogamente, pode-se mostrar que $\mdc(a,b) = \mdc(a, b+ac)$. 
	\end{proof}
	\begin{exercise}
		Mostre que se $\mdc(a,b) = 1$, então $\mdc(a+b, ab) = 1$.
	\end{exercise}
	\begin{proof}
		Por definição, temos que $d = \mdc(a+b, ab)$ é um número natural que divide $a+b$ e $ab$. Como $d|ab$, segue que $d|a$ ou $d|b$, pois a única maneira de $d$ \textbf{não} dividir $ab$ é se $d$ não divide nem $a$ nem $b$ (basta fazer as contas usando o Algoritmo da Divisão de Euclides). Sem perda de generalidade, suponha que $d|a$. Daí, segue que existem $k_1, k_2\in\mathbb{Z}$ tais que
		\begin{align*}
		\begin{cases}
		a + b = k_1d \\
		a = k_2d 
		\end{cases} \Rightarrow b = (k_1 - k_2)d \Leftrightarrow d|b
		\end{align*}
		\par Portanto, se $d|a$ então $d|b$ e vice versa, ou seja, $d|a$ e $d|b$. Consequentemente, $d$ é divisor comum de $a$ e $b$, ou seja, $d\leq \mdc(a,b) = 1$. Como $d\in\mathbb{N}$, segue que $d=1$.
	\end{proof}
	\begin{exercise}
		Mostre que $\mdc(3m + 2, 5m+3) = 1$.
	\end{exercise}
	\begin{proof}
		Vamos usar o Exercício 3. Temos que
		\begin{align*}
			\mdc(3m+2, 5m+3) &= \mdc( 3m+2, 2m+1 ) \\
			&= \mdc(m + 1, 2m+1) \\
			&= \mdc(m+1, m) \\
			&= \mdc(1, m) \\
			&= \mdc(1,1) \\
			&= 1
		\end{align*}
		\par Note que de $\mdc(m+1, m)$ já poderíamos concluir que o $\mdc$ é $1$, pois a menor combinação linear (ou seja, o $mdc$) de $m+1$ e $m$ é justamente 
		\begin{align*}
		1(m+1) -1(m) = 1
		\end{align*}
	\end{proof}
	\begin{exercise}
		Use o algoritmo de Euclides para calcular $d = \mdc(a,b)$, e encontrar $r,s\in\mathbb{Z}$ tais que $d = ra + sb$.
		\begin{equation*}
		a=412, b=32 \text{; } a=780, b=150 \text{; } a=10672, b=414
		\end{equation*}
	\end{exercise}
	\begin{solution}
		Para $a = 412$ e $b= 32$, temos
		\begin{align*}
		412 &= 32\cdot 12 + 28 \\
		32 &= 28\cdot 1 + 4 \\
		28 &= 4\cdot 7 
		\end{align*}
		\par Portanto, $\mdc(412, 32) = 4$. Reescrevendo os restos, temos
		\begin{align*}
		28 &= 412 - 32\cdot 12 \\
		4 &= 32 - 28 = 32 - (412 - 12\cdot 32) = -412 + 13\cdot 32 \therefore (r,s) = (-1,13)
		\end{align*}
		Para $a = 780$ e $b= 150$, temos
		\begin{align*}
		780 &= 150\cdot 5 + 30 \\
		150 &= 30\cdot 5 
		\end{align*}
		\par Portanto, $\mdc(780, 150) = 30$. Reescrevendo os restos, temos
		\begin{align*}
		30 &= 780 - 5\cdot 150 \therefore (r,s) = (1,-5)
		\end{align*}
		Para $a = 10672$ e $b= 414$, temos
		\begin{align*}
		10672 &= 414\cdot 25 + 322 \\
		414 &= 322\cdot 1 + 92 \\
		322 &= 92\cdot 3 + 46 \\
		92 &= 46\cdot 2 
		\end{align*}
		\par Portanto, $\mdc(10672, 414) = 46$. Reescrevendo os restos, temos
		\begin{align*}
		322 &= 10672 - 414\cdot 25 \\
		92 &= 414 - 322 = 414 - (10672 - 25\cdot 414) = -10672 + 26\cdot 414 \\
		46 &= 322 - 3\cdot 92 = 10672 - 25\cdot 414 - 3( -10672 + 26\cdot 414 ) = 4\cdot 10672 - 103\cdot 414 
		\\ &\therefore (r,s) = (4,-103)
		\end{align*}
		
		
		
	\end{solution}

	
\end{document}
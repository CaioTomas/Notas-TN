\documentclass[a4paper,12pt]{article}
\usepackage[utf8]{inputenc}
\usepackage{graphicx}
\usepackage{amsmath}
\usepackage{amssymb}
\usepackage{amsfonts}
\usepackage{textcomp}
\usepackage{amsthm}
\usepackage{subcaption}
\usepackage{booktabs}
\usepackage{float}
\usepackage{mathrsfs}
\usepackage{mathtools}
\usepackage{xfrac}
\usepackage{hyperref}
\usepackage{xfrac}
\usepackage{enumerate}
\usepackage{cancel}
\DeclareMathOperator{\tr}{tr}
\DeclareMathOperator{\aut}{Aut}
\DeclareMathOperator{\inn}{Inn}
\DeclareMathOperator{\stab}{stab}
\DeclareMathOperator{\orb}{orb}
\DeclareMathOperator{\Mod}{Mod}
\DeclareMathOperator{\Ker}{Ker}
\DeclareMathOperator{\mdc}{mdc}
\DeclareMathOperator{\mmc}{mmc}
\DeclareMathOperator{\syl}{Syl}
\DeclareMathOperator{\tor}{Tor}
\DeclareMathOperator{\Sim}{Sim}

% Times for rm and math | Helvetica for ss | Courier for tt
%\usepackage{mathptmx} % rm & math
%\usepackage[scaled=0.90]{helvet} % ss
%\usepackage{courier} % tt
%\normalfont
%\usepackage[T1]{fontenc}

\usepackage[all]{xy}

%\input xy
%\xyoption{all}

%\documentclass{standalone}

\usepackage{anonchap}
\usepackage[symbol]{footmisc}
\usepackage{anonchap}
\usepackage[Sonny]{fncychap}
%\usepackage[brazilian]{babel}
%\usepackage[portuguese]{babel}
\usepackage{geometry}
\geometry{a4paper, left=3cm, top=3cm, right=2cm, bottom=2cm}
\usepackage{multicol}
\usepackage{fancyhdr}
%\usepackage[center]{caption}

\theoremstyle{definition}
\newtheorem{theorem}{Teorema}[section]
\newtheorem*{definition}{Definição}
\newtheorem{corollary}{Corolário}[theorem]
\newtheorem{lemma}[theorem]{Lema}
\newtheorem{remark}{Observação}[section]
\newtheorem{deff}{Definição}[section]
\newtheorem{fact}{Fato}[section]
\newtheorem{exercise}{Exercício}%[section]
\newtheorem{example}{Exemplo}[section]
\newtheorem{prop}{Proposição}[section]
\newtheorem*{solution}{Solução}
\title{Resolução Lista 2}
\date{}
\author{Caio Tomás}
\begin{document}
	\maketitle
	\subsubsection*{Equações diofantinas lineares}
	\begin{exercise}
		Calculando $\mdc(512, 28)$ usando o Algoritmo de Euclides para o cálculo do MDC, segue que
		\begin{align*}
			512 &= 28\cdot 18 + 8 \\
			28 &= 8\cdot 3 + 4 \\
			8 &= 4\cdot 2
		\end{align*}
		Logo, $\mdc(512,28) = 4$. Como $4|48$, segue que a equação diofantina $512x + 28y = 48$ tem infinitas soluções. Reescrevendo os restos acima, temos
		\begin{align*}
		8 &= 512 - 28\cdot 18 \\
		4 &= 28 - 3\cdot 8 = 28 - 3\cdot( 512 - 28\cdot 18 ) = (-3)\cdot 512 + (55)\cdot 28
		\end{align*}
		Portanto, segue que
		\begin{align*}
		48 = (-36)\cdot 512 + (660)\cdot 28
		\end{align*}
		Portanto, uma solução é $x_0, y_0 = -36, 660$. A partir daí, pelo Lema 8, as outras soluções são dadas por
		\begin{align*}
		\begin{cases}
		x = -36 + \displaystyle{\frac{28}{4}}\cdot t \\
		y = 660 - \displaystyle{\frac{512}{4}}\cdot t
		\end{cases} \Leftrightarrow \begin{cases}
		x = -36 + 7t \\
		y = 660 - 128t
		\end{cases}, t\in\mathbb{Z}
		\end{align*}
	\end{exercise}
	\begin{exercise}
		Calculando $\mdc(1257, 108)$ usando o Algoritmo de Euclides para o cálculo do MDC, temos
		\begin{align*}
		1257 &= 108\cdot 11 + 69 \\
		108 &= 69\cdot 1 + 39 \\
		69 &= 39\cdot 1 + 30 \\
		39 &= 30\cdot 1 + 9 \\
		30 &= 9\cdot 3 + 3 \\
		9 &= 3\cdot 3 
		\end{align*}
		Logo, $\mdc(1257, 108) = 3$. Como $3|54$, segue que a equação diofantina $1257x + 108y = 54$ tem infinitas soluções. Reescrevendo os restos acima, temos
		\begin{align*}
		69 &= 1257 - 108\cdot 11 \\
		39 &= 108 - 69 = -1257 + 12\cdot 108 \\
		30 &= 69 - 39 = 1257 - 11\cdot 108 + 1257 - 12\cdot 108 = 2\cdot 1257 - 23\cdot 108 \\
		9 &= 39 - 30 = -1257 + 12\cdot 108 - 2\cdot 1257 + 23\cdot 108 = -3\cdot 1257 + 35\cdot 108 \\
		3 &= 30 - 3\cdot 9 = 2\cdot 1257 - 23\cdot 108 + 9\cdot 1257 - 105\cdot 108 = (11)\cdot 1257 + (-128)\cdot 108
		\end{align*}
		Portanto, segue que
		\begin{align*}
		54 = (198)\cdot 1257 + (-2304)\cdot 108
		\end{align*}
		Portanto, uma solução é $x_0, y_0 = 198, -2304$. A partir daí, pelo Lema 8, as outras soluções são dadas por
		\begin{align*}
		\begin{cases}
		x = 198 + \displaystyle{\frac{108}{3}}\cdot t \\
		y = -2304 - \displaystyle{ \frac{1257}{3}}\cdot t
		\end{cases} \Leftrightarrow \begin{cases}
		x = 198 + 36t \\
		y = -2304 - 419t
		\end{cases}, t\in\mathbb{Z}
		\end{align*}		
	\end{exercise}
	\begin{exercise}
		Calculando $\mdc(4875, 2223)$ usando o Algoritmo de Euclides para o cálculo do MDC, temos
		\begin{align*}
		4875 &= 2223\cdot 2 + 429 \\
		2223 &= 429\cdot 5 + 78 \\
		429 &= 78\cdot 5 + 39 \\
		78 &= 39\cdot 2 
		\end{align*}
		Logo, $\mdc(4875, 2223) = 39$. Como $39|117$, segue que a equação diofantina $4875x + 2223y = 117$ tem infinitas soluções. Reescrevendo os restos acima, temos
		\begin{align*}
		429 &= 4875 - 2\cdot 2223 \\
		78 &= 2223 - 5\cdot 429 = -5\cdot 4875 + 11\cdot 2223 \\
		39 &= 429 - 5\cdot 78 = (26)\cdot 4875 + (-57)\cdot 2223
		\end{align*}
		Portanto, segue que 
		\begin{align*}
		117 = (78)\cdot 4875 + (-171)\cdot 2223
		\end{align*}
		Portanto, uma solução é $x_0, y_0 = 78, -171$. A partir daí, pelo Lema 8, as outras soluções são dadas por
		\begin{align*}
		\begin{cases}
		x = 78 + \displaystyle{ \frac{2223}{39} }\cdot t \\
		y = -171 - \displaystyle{ \frac{4875}{39} }\cdot t
		\end{cases} \Leftrightarrow \begin{cases}
		x = 78 + 57t \\
		y = -171 - 125t
		\end{cases}, t\in\mathbb{Z}
		\end{align*}
	\end{exercise}
	\subsubsection*{Indução matemática}
	\begin{exercise}
		\begin{proof}
		Para o caso particular $n = 1$, temos
		\begin{align*}
		1 = 1^2
		\end{align*}
		logo $p(1)$ é verdadeira. Suponha, por hipótese de indução, que 
		\begin{align*}
		1 + 3 + \cdots + (2k-1) = k^2
		\end{align*}
		e considere
		\begin{align*}
		1 + 3 + \cdots + (2k - 1) + (2k + 1) = k^2 + 2k + 1 = (k + 1)^2
		\end{align*}
		ou seja, $p(k+1)$ é verdadeira, pois $2k + 1 = 2(k+1) - 1$. Logo, por indução, segue que $1 + 3 + \cdots + (2n-1) = n^2, \forall n\in\mathbb{N}$.
	\end{proof}
	\end{exercise}
	\begin{exercise}
		\begin{proof}
			Para o caso particular $n=1$, temos
			\begin{align*}
				1^2 = \frac{1\cdot 2\cdot 3}{6} = 1
			\end{align*}
			logo $p(1)$ é verdadeira. Suponha, por hipótese de indução, que 
			\begin{align*}
			1^2 + \cdots + k^2 = \frac{k(k+1)(2k+1)}{6}
			\end{align*}
			e considere
			\begin{align*}
			1^2 + \cdots + k^2 + (k+1)^2 &= \frac{k(k+1)(2k+1)}{6} + (k+1)^2 \\
			&= \frac{ (k+1)(2k^2+k+6k+6) }{ 6 } \\
			&= \frac{(k+1)(2k^2 + 7k + 6)  }{6} \\
			&= \frac{(k+1)(k+2)(2k+3) }{6}
			\end{align*}
			ou seja, $p(k+1)$ é verdadeira, pois $2k + 3 = 2(k+1) + 1$. Logo, por indução, segue que $1^2 + 2^2 + \cdots + n^2 = \displaystyle{ \frac{n(n+1)(2n+1)}{6} }, \forall n\in\mathbb{N}$.
		\end{proof}
	\end{exercise}
	\begin{exercise}
		\begin{proof}
			Para o caso particular $n=1$, temos
			\begin{align*}
			1\cdot 2 = \frac{1\cdot 2\cdot 3}{3} = 2
			\end{align*}
			logo $p(1)$ é verdadeira. Suponha, por hipótese de indução, que 
			\begin{align*}
			1\cdot 2 + 2\cdot 3 + \cdots + k\cdot(k+1) = \frac{k(k+1)(k+2)}{3}
			\end{align*}
			e considere
			\begin{align*}
			1\cdot 2 + 2\cdot 3 + \cdots + k\cdot (k+1) + (k+1)\cdot (k+2) &= \frac{k(k+1)(k+2)}{3} + (k+1)\cdot (k+2) \\
			&= \frac{(k+1)(k+2)(k+3)}{3}
			\end{align*}
			ou seja, $p(k+1)$ é verdadeira. Logo, por indução, segue que $1\cdot 2 + \cdots + n\cdot (n+1) = \displaystyle{\frac{n(n+1)(n+2)}{3}}, \forall n\in\mathbb{N}$.
		\end{proof}
	\end{exercise}
	\begin{exercise}
		\begin{proof}
			Para o caso particular $n=4$, temos
			\begin{align*}
			2^4 = 16 < 24 = 4!
			\end{align*}
			logo $p(4)$ é verdadeira. Suponha, por hipótese de indução, que 
			\begin{align*}
			2^k < k!
			\end{align*} 
			e multiplique ambos os lados da igualdade por $k+1$ para obter
			\begin{align*}
			2^k(k+1) < (k+1)!
			\end{align*}
			Agora, note que como $k\geq 4$ (por hipótese), então $k+1\geq 5$. Daí, temos
			\begin{align*}
			2^{k+1} < 2^k\cdot 5 \leq 2^k(k+1) < (k+1)!
			\end{align*}
			ou seja, $p(k+1)$ é verdadeira. Logo, por indução, segue que $2^n < n!, \forall n\geq 4$.
		\end{proof}
	\end{exercise}
	\begin{exercise}
		\begin{proof}
			Para o caso particular $n=1$, temos
			\begin{align*}
			\frac{2!}{2\cdot 1!} = 1\in\mathbb{Z}
			\end{align*}
			logo $p(1)$ é verdadeira. Suponha, por hipótese de indução, que
			\begin{align*}
			\frac{(2k)!}{2^k\cdot k!}\in\mathbb{Z}
			\end{align*} 
			e considere
			\begin{align*}
			\frac{(2k+2)!}{2^{k+1}\cdot (k+1)!} = \frac{ (2k+2)(2k+1) }{2(k+1) }\cdot\underbrace{\frac{(2k)!}{2^k\cdot k!}}_{\in\mathbb{Z}} = (2k+1)\cdot\underbrace{\frac{(2k)!}{2^k\cdot k!}}_{\in\mathbb{Z}}\in\mathbb{Z}
			\end{align*}
			pois $2k+1\in\mathbb{Z}$. Portanto, $p(k+1)$ é verdadeira, e segue por indução que $\displaystyle{ \frac{(2k)!}{2^k\cdot k!}\in\mathbb{Z},\forall n\in\mathbb{N} }$. 
		\end{proof}
	\end{exercise}
	
	
	
\end{document}
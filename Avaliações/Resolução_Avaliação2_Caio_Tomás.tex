\documentclass[a4paper,12pt]{article}
\usepackage[utf8]{inputenc}
\usepackage{graphicx}
\usepackage{amsmath}
\usepackage{amssymb}
\usepackage{amsfonts}
\usepackage{textcomp}
\usepackage{amsthm}
\usepackage{subcaption}
\usepackage{booktabs}
\usepackage{float}
\usepackage{mathrsfs}
\usepackage{mathtools}
\usepackage{xfrac}
\usepackage{hyperref}
\usepackage{xfrac}
\usepackage{enumerate}
\usepackage{cancel}
\DeclareMathOperator{\tr}{tr}
\DeclareMathOperator{\aut}{Aut}
\DeclareMathOperator{\inn}{Inn}
\DeclareMathOperator{\stab}{stab}
\DeclareMathOperator{\orb}{orb}
\DeclareMathOperator{\Mod}{Mod}
\DeclareMathOperator{\Ker}{Ker}
\DeclareMathOperator{\mdc}{mdc}
\DeclareMathOperator{\mmc}{mmc}
\DeclareMathOperator{\syl}{Syl}
\DeclareMathOperator{\tor}{Tor}
\DeclareMathOperator{\Sim}{Sim}

% Times for rm and math | Helvetica for ss | Courier for tt
%\usepackage{mathptmx} % rm & math
%\usepackage[scaled=0.90]{helvet} % ss
%\usepackage{courier} % tt
%\normalfont
%\usepackage[T1]{fontenc}

\usepackage[all]{xy}

%\input xy
%\xyoption{all}

%\documentclass{standalone}

\usepackage{anonchap}
\usepackage[symbol]{footmisc}
\usepackage{anonchap}
\usepackage[Sonny]{fncychap}
%\usepackage[brazilian]{babel}
%\usepackage[portuguese]{babel}
\usepackage{geometry}
\geometry{a4paper, left=3cm, top=3cm, right=2cm, bottom=2cm}
\usepackage{multicol}
\usepackage{fancyhdr}
%\usepackage[center]{caption}

\theoremstyle{definition}
\newtheorem{theorem}{Teorema}[section]
\newtheorem*{definition}{Definição}
\newtheorem{corollary}{Corolário}[theorem]
\newtheorem{lemma}[theorem]{Lema}
\newtheorem{remark}{Observação}[section]
\newtheorem{deff}{Definição}[section]
\newtheorem{fact}{Fato}[section]
\newtheorem{exercise}{Exercício}%[section]
\newtheorem{example}{Exemplo}[section]
\newtheorem{prop}{Proposição}[section]
\newtheorem*{solution}{Solução}
\title{Resolução Lista de Avaliação 2}
\date{18/09/2020}
\author{Caio Tomás de Paula}
\begin{document}
	\maketitle
	\subsection*{Questão 1.}
	\begin{proof}
	Pelo TFA, temos 
	\begin{align*}
	c = q_1q_2\cdots q_s, \text{ com } q_1 < q_2 < \cdots < q_s \text{ primos.}
	\end{align*}
	Daí, é claro que
	\begin{align*}
	c^m = q_1^mq_2^m\cdots q_s^m
	\end{align*}
	ou seja, elevar $c$ à $m$-ésima potência não altera os primos que aparecem na fatoração de $c$, apenas sua quantidade. Desse modo, se $p|c^m$, então necessariamente $p=q_j$ para algum $1\leq j\leq s$. Sem perda de generalidade, suponha que $p = q_1$. Desse modo, temos
	\begin{align*}
	c^m = p^m(q_2^m\cdots q_s^m)
	\end{align*}
	Note, por fim, que como $p^{m+1}\nmid c^m$, então, pelo Lema 9, $\mdc(p^{m+1}, c^m) = 1$. Em outras palavras, a maior potência de $p$ que aparece na fatoração de $c^m$ é $p^m$. Isso nos diz que em $\lambda = q_2^m\cdots q_s^m$ teremos $p\neq q_i$, para todo $2\leq i\leq s$. Dito de outro modo, temos
	\begin{align*}
	c^m = p^m\lambda, \text{ com } \mdc(p,\lambda) = 1
	\end{align*}
	como queríamos.
\end{proof}
\subsection*{Questão 2.}
\begin{proof}
	Note que
	\begin{align*}
	n^3 + 1 = \underbrace{(n+1)}_{\geq 3}\underbrace{(n^2 - n + 1)}_{\geq 3} 
	\end{align*}
	ou seja, $n^3 + 1$ pode ser escrito como um produto de dois inteiros maiores que $1$. Em particular, segue do TFA que $n^3 + 1$ pode ser escrito como o produto de pelo menos dois primos, sendo, portanto, composto.
\end{proof}
\subsection*{Questão 3.}
\begin{proof}
	Vamos proceder por indução em $n$. Como casos particulares, note que
	\begin{align*}
	&F_1 = 1 = F_2 \\
	&F_1 + F_3 = 1 + 2 = 3 = F_4
	\end{align*}
	e nossa proposição é verdadeira. Suponha, por hipótese de indução, que 
	\begin{align*}
	F_1 + F_3 + \cdots + F_{2k-1} = F_{2k}
	\end{align*}
	e considere
	\begin{align}
	\label{tese inducao}
	F_1 + F_3 + \cdots + F_{2k-1} + F_{2k+1}
	\end{align}
	Usando a hipótese de indução em \eqref{tese inducao}, obtemos
	\begin{align*}
	F_1 + F_3 + \cdots + F_{2k-1} + F_{2k+1} &= F_{2k} + F_{2k+1} \\
	&= F_{2(k+1)}
	\end{align*}
	pela recorrência da sequência de Fibonacci, como desejado, e o resultado segue por indução.
\end{proof}
	
	
	
	
	
	
	
	
	
	
	
	
	
	
\end{document}
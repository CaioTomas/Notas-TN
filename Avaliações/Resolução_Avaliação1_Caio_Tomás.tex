\documentclass[a4paper,12pt]{article}
\usepackage[utf8]{inputenc}
\usepackage{graphicx}
\usepackage{amsmath}
\usepackage{amssymb}
\usepackage{amsfonts}
\usepackage{textcomp}
\usepackage{amsthm}
\usepackage{subcaption}
\usepackage{booktabs}
\usepackage{float}
\usepackage{mathrsfs}
\usepackage{mathtools}
\usepackage{xfrac}
\usepackage{hyperref}
\usepackage{xfrac}
\usepackage{enumerate}
\usepackage{cancel}
\DeclareMathOperator{\tr}{tr}
\DeclareMathOperator{\aut}{Aut}
\DeclareMathOperator{\inn}{Inn}
\DeclareMathOperator{\stab}{stab}
\DeclareMathOperator{\orb}{orb}
\DeclareMathOperator{\Mod}{Mod}
\DeclareMathOperator{\Ker}{Ker}
\DeclareMathOperator{\mdc}{mdc}
\DeclareMathOperator{\mmc}{mmc}
\DeclareMathOperator{\syl}{Syl}
\DeclareMathOperator{\tor}{Tor}
\DeclareMathOperator{\Sim}{Sim}

% Times for rm and math | Helvetica for ss | Courier for tt
%\usepackage{mathptmx} % rm & math
%\usepackage[scaled=0.90]{helvet} % ss
%\usepackage{courier} % tt
%\normalfont
%\usepackage[T1]{fontenc}

\usepackage[all]{xy}

%\input xy
%\xyoption{all}

%\documentclass{standalone}

\usepackage{anonchap}
\usepackage[symbol]{footmisc}
\usepackage{anonchap}
\usepackage[Sonny]{fncychap}
%\usepackage[brazilian]{babel}
%\usepackage[portuguese]{babel}
\usepackage{geometry}
\geometry{a4paper, left=3cm, top=3cm, right=2cm, bottom=2cm}
\usepackage{multicol}
\usepackage{fancyhdr}
%\usepackage[center]{caption}

\theoremstyle{definition}
\newtheorem{theorem}{Teorema}[section]
\newtheorem*{definition}{Definição}
\newtheorem{corollary}{Corolário}[theorem]
\newtheorem{lemma}[theorem]{Lema}
\newtheorem{remark}{Observação}[section]
\newtheorem{deff}{Definição}[section]
\newtheorem{fact}{Fato}[section]
\newtheorem{exercise}{Exercício}%[section]
\newtheorem{example}{Exemplo}[section]
\newtheorem{prop}{Proposição}[section]
\newtheorem*{solution}{Solução}
\title{Resolução Lista de Avaliação 1}
\date{04/09/2020}
\author{Caio Tomás de Paula}
\begin{document}
	\maketitle
	\subsection*{Questão 1.}
	\begin{proof}
	Pelo Teorema Fundamental da Aritmética e pelo Lema 15, sabemos que
	\begin{align*}
	a = \prod_{i=1}^{n}p_i^{\alpha_i}, \text{  }
	b = \prod_{i=1}^{n}p_i^{\beta_i}, \text{  }
	c = \prod_{i=1}^{n}p_i^{\gamma_i}
	\end{align*}
	com $p_1\leq p_2\leq \cdots \leq p_n$ primos e $0\leq \alpha_i, \beta_i \leq \gamma_i$, pois $a|c$ e $b|c$. Como $\mdc(a,b) = 1$, segue que $a$ e $b$ não compartilham fatores primos, ou seja, $\alpha_i = 0$ sempre que $\beta_i\neq 0$. Como $a|c$ e $b|c$, sabemos que $c = \lambda_1a = \lambda_2b$, com $\lambda_1, \lambda_2\in\mathbb{Z}$, ou seja, $\lambda_1a = \lambda_2b$. Como $a$ e $b$ não compartilham fatores primos, então devemos ter, necessariamente, $\lambda_2 = n_2a$ e $\lambda_1 = n_1b$, $n_1, n_2\in\mathbb{Z}$. Substituindo, segue que $c = n_1ba = n_2ab$, ou seja, $ab|c$.
\end{proof}
\subsection*{Questão 2.}
\begin{proof}
	Seja $m\in\mathbb{Z}$ ímpar. Podemos escrever $m = 2j + 1$, $j\in\mathbb{Z}$. Elevando ao quadrado, obtemos
	\begin{align*}
	m^2 = 4j^2 + 4j + 1
	\end{align*}
	Temos duas possibilidades: ou $j$ é par, ou $j$ é ímpar. Se $j$ é par, então podemos escrever $j = 2n, n\in\mathbb{Z}$. Daí, segue que
	\begin{align*}
	m^2 = 4(2n)^2 + 4(2n) + 1 = 8\underbrace{(2n^2 + n)}_{k\in\mathbb{Z}} + 1
	\end{align*}
	Se $j$ é ímpar, podemos escrever $j = 2n+1, n\in\mathbb{Z}$. Daí, segue que
	\begin{align*}
	m^2 = 4(2n+1)^2 + 4(2n + 1) + 1 = 4(4n^2 + 4n + 1) + 8n + 4 + 1 =8\underbrace{(2n^2 + 3n + 1)}_{k\in\mathbb{Z}} + 1
	\end{align*}
\end{proof}
\subsection*{Questão 3.}
\begin{proof}
	Usando o algoritmo de Euclides, temos
	\begin{align*}
	1350 &= 2\cdot 504 + 342 \\
	504 &= 1\cdot 342 + 162 \\
	342 &= 2\cdot 162 + 18 \\
	162 &= 9\cdot 18
	\end{align*}
	Portanto, segue que $d = 18$. Como $d=18|54$, segue dos Lemas 7 e 8 que a equação diofantina $1350x + 504y = 54$ tem infinitas soluções. Reescrevendo os restos acima, obtemos
	\begin{align*}
	342 &= 1350 - 2\cdot 504 \\
	162 &= 504 - 342 = -1350 + 3\cdot 504 \\
	18 &= 342 - 2\cdot 162 = 3\cdot 1350 + (-8)\cdot 504
	\end{align*}
	Consequentemente, temos $r,s = 3, -8$. Multiplicando a última igualdade por $3$, obtemos
	\begin{align*}
	54 = (9)\cdot 1350 + (-24)\cdot 504
	\end{align*}
	Portanto, $x_0, y_0 = 9, -24$ é solução de $1350x + 504y = 54$. Segue do Lema 8 que as outras soluções têm a forma
	\begin{align*}
	\begin{cases}
	x = 9 + 28t \\
	y = -24 - 75t
	\end{cases}, t\in\mathbb{Z}
	\end{align*}
\end{proof}
	
\subsection*{Questão 4.}
\begin{proof}
	Para o caso particular $n=1$, temos
	\begin{align*}
	1^2 = \frac{1\cdot 1\cdot 3}{3} = 1
	\end{align*}
	Suponha, por hipótese de indução, que 
	\begin{align*}
	1^2 + 3^2 + \cdots + (2k-1)^2 = \frac{k(2k-1)(2k+1)}{3}
	\end{align*}
	Agora, considere
	\begin{align*} 
	1^2 + 3^2 + \cdots + (2k-1)^2 + (2k+1)^2 &= \frac{k(2k-1)(2k+1)}{3} + (2k+1)^2 \\
	&= \frac{k(2k-1)(2k+1) + 3(2k+1)^2}{3} \\
	&= \frac{(2k+1)(k(2k-1) + 3(2k+1)}{3} \\
	&= \frac{(2k+1)(2k^2 + 5k + 3)}{3} \\
	&= \frac{(2k+1)(k+1)(2k + 3)}{3} \\
	&= \frac{(k+1)(2k+1)(2k+3)}{3}
	\end{align*}
	Daí, segue por indução que $1^2 + 3^2 + \cdots + (2n-1)^2 = \displaystyle{ \frac{n(2n-1)(2n+1)}{3} }.$
\end{proof}
	
	
	
	
	
	
	
	
	
	
	
	
	
	
\end{document}
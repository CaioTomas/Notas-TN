\documentclass[a4paper,12pt]{article}
\usepackage[utf8]{inputenc}
\usepackage{graphicx}
\usepackage{amsmath}
\usepackage{amssymb}
\usepackage{amsfonts}
\usepackage{textcomp}
\usepackage{amsthm}
\usepackage{subcaption}
\usepackage{booktabs}
\usepackage{float}
\usepackage{mathrsfs}
\usepackage{mathtools}
\usepackage{xfrac}
\usepackage{hyperref}
\usepackage{xfrac}
\usepackage{enumerate}
\usepackage{cancel}
\DeclareMathOperator{\tr}{tr}
\DeclareMathOperator{\aut}{Aut}
\DeclareMathOperator{\inn}{Inn}
\DeclareMathOperator{\stab}{stab}
\DeclareMathOperator{\orb}{orb}
\DeclareMathOperator{\Mod}{Mod}
\DeclareMathOperator{\Ker}{Ker}
\DeclareMathOperator{\mdc}{mdc}
\DeclareMathOperator{\mmc}{mmc}
\DeclareMathOperator{\syl}{Syl}
\DeclareMathOperator{\tor}{Tor}
\DeclareMathOperator{\Sim}{Sim}

% Times for rm and math | Helvetica for ss | Courier for tt
%\usepackage{mathptmx} % rm & math
%\usepackage[scaled=0.90]{helvet} % ss
%\usepackage{courier} % tt
%\normalfont
%\usepackage[T1]{fontenc}

\usepackage[all]{xy}

%\input xy
%\xyoption{all}

%\documentclass{standalone}

\usepackage{anonchap}
\usepackage[symbol]{footmisc}
\usepackage{anonchap}
\usepackage[Sonny]{fncychap}
%\usepackage[brazilian]{babel}
%\usepackage[portuguese]{babel}
\usepackage{geometry}
\geometry{a4paper, left=3cm, top=3cm, right=2cm, bottom=2cm}
\usepackage{multicol}
\usepackage{fancyhdr}
%\usepackage[center]{caption}

\theoremstyle{definition}
\newtheorem{theorem}{Teorema}[section]
\newtheorem*{definition}{Definição}
\newtheorem{corollary}{Corolário}[theorem]
\newtheorem{lemma}[theorem]{Lema}
\newtheorem{remark}{Observação}[section]
\newtheorem{deff}{Definição}[section]
\newtheorem{fact}{Fato}[section]
\newtheorem{exercise}{Exercício}%[section]
\newtheorem{example}{Exemplo}[section]
\newtheorem{prop}{Proposição}[section]
\newtheorem*{solution}{Solução}
\title{Resolução Lista de Avaliação 3}
\date{30/10/2020}
\author{Caio Tomás de Paula}
\begin{document}
	\maketitle
	\subsection*{Questão 1.}
	\begin{proof}
		Pelo Lema de Euler, temos
		\begin{align*}
		619^{\phi(15)}\equiv 1\mod 15 \Leftrightarrow 619^{8}\equiv 1\mod 15
		\end{align*}
		Note que $1367 = 8\cdot 17 + 7$, logo
		\begin{align*}
		619^{1367} = (619^{8})^{17}\cdot 619^7 \equiv 619^7 \equiv 4^7 \equiv (4^2)^3\cdot 4 \equiv 4 \mod 15
		\end{align*}
		e o resto da divisão de $619^{1367}$ por $15$ é $4$.
	\end{proof}
	\subsection*{Questão 2.}
	\begin{proof}
		Escreva
		\begin{align*}
		n = 2^{a}\cdot p_2^{\alpha_2}\cdots p_{k+1}^{\alpha_{k+1}}
		\end{align*}
		com $p_2, \dots, p_{k+1}$ primos ímpares distintos, $a\in\mathbb{N}\cup\left\{0\right\}$ e $\alpha_i\in\mathbb{N}, 2\leq i\leq k+1$. Sabemos que
		\begin{align*}
		\phi(n) = 2^{a-1}\cdot p_2^{\alpha_2-1}\cdots p_{k+1}^{\alpha_{k+1}-1}\cdot (2-1)\prod_{i=2}^{k+1}(p_i - 1).
		\end{align*}
		Como os $p_i$'s são ímpares, segue que os $p_i-1$'s são pares, de modo que há pelo menos $k$ fatores $2$ em $\displaystyle{\prod_{i=2}^{k+1} (p_i - 1)}$, ou seja
		\begin{align*}
		2^k\Big|\prod_{i=2}^{k+1}(p_i-1)
		\end{align*}
		e portanto
		\begin{align*}
		2^k|\phi(n).
		\end{align*}
	\end{proof}
	\subsection*{Questão 3.}
	\begin{proof}
		Note que
		\begin{align*}
		4n^2 + 4 = 4(n^2 + 1)
		\end{align*}
		e, portanto, se $19|4n^2+4$ então $19|n^2 + 1$, ou seja,
		\begin{align*}
		n^2 + 1\equiv 0\mod 19 \Leftrightarrow n^2\equiv 18\mod 19 \Leftrightarrow n^{18}\equiv 18^9\mod 19.
		\end{align*}
		Sabemos que $1$ e $18$ são os únicos elementos em $\left\{ 1,2,\dots,18 \right\}$ cujos inversos são eles próprios e, portanto,
		\begin{align*}
		18^2\equiv 1\mod 19\Leftrightarrow 18^8\equiv 1\mod 19\Leftrightarrow 18^9\equiv 18\mod 19.
		\end{align*}
		Desse modo, temos que
		\begin{align*}
		n^{18}\equiv 18^9\equiv 18\mod 19
		\end{align*}
		Por outro lado, temos por Euler que
		\begin{align*}
		n^{18}\equiv 1\mod 19
		\end{align*}
		o que é absurdo! Logo, $19\nmid 4n^2+4, \forall n\in\mathbb{N}$. 
	\end{proof}
	\subsection*{Questão 4.}
	\begin{proof}
		Vamos utilizar o Teorema do Resto Chinês para resolver
		\begin{align*}
		\begin{cases}
		2x\equiv 3\mod 5 \\
		3x\equiv 2\mod 7 \\
		4x\equiv 6\mod 9 \\
		6x\equiv 4\mod 11
		\end{cases}
		\end{align*} 
		pois as hipóteses são satisfeitas.
		\begin{enumerate}[(i)]
			\item Soluções individuais
			\begin{align*}
			x_1 = 4, x_2 = 3, x_3 = 6, x_4 = 8
			\end{align*}
			\item Cálculo dos $n_i$'s
			\begin{align*}
			n_1 = 7\cdot 9\cdot 11 = 693, n_2 = 5\cdot 9\cdot 11 = 495, n_3 = 5\cdot 7\cdot 11 = 385, n_4 = 5\cdot 7\cdot 9 = 315
			\end{align*}
			\item Cálculo dos $z_i$'s
			\begin{align*}
			693z\equiv 1\mod 5 \Rightarrow 3z\equiv 1\mod 5 \Rightarrow z_1 = 2 \\
			495z\equiv 1\mod 7 \Rightarrow 5z\equiv 1\mod 7 \Rightarrow z_2 = 3 \\
			385z\equiv 1\mod 9 \Rightarrow 7z\equiv 1\mod 9 \Rightarrow z_3 = 4 \\
			315z\equiv 1\mod 11 \Rightarrow 7z\equiv 1\mod 11 \Rightarrow z_4 = 8 \\
			\end{align*}
			\item Solução geral
			\begin{align*}
			x_0 &= 4\cdot 693\cdot 2 + 3\cdot 495\cdot 3 + 6\cdot 385\cdot 4 + 8\cdot 315\cdot 8 \\
			&= 39399 \\
			&\equiv 1284\mod 3465 (= 5\cdot 7\cdot 9\cdot 11)
			\end{align*}
		\end{enumerate}
	\end{proof}
	\subsection*{Questão 5.}
	\begin{proof}
		Vamos utilizar o lema de Hensel. Note que
		\begin{align*}
		f(x) &= x^4 + 10x^2 + x + 3 \\
		f'(x) &= 4x^3 + 20x + 1
		\end{align*}
		Note que 
		\begin{align*}
		f(1) = 15 = 5\cdot 3\equiv 0\mod 3 \\
		f'(1) = 25\equiv 1\not\equiv 0\mod 3
		\end{align*}
		e podemos aplicar o lema de Hensel. Então, queremos $t_1$ tal que
		\begin{align*}
		f(1 + 3t_1) \equiv 0\mod 3^2 \Leftrightarrow 5 + t_1\equiv 0\mod 3\Leftrightarrow t_1\equiv 1\mod 3.
		\end{align*}
		Logo, $a_1 = 1 + 3 = 4$. Note que
		\begin{align*}
		f(4) &= 423 = 47\cdot 9\equiv 0\mod 9 \\
		f'(4) &= 337 \equiv 1\not\equiv 0\mod 3
		\end{align*}
		de modo que podemos aplicar o lema de Hensel novamente. Então, queremos $t_2$ tal que
		\begin{align*}
		f(4+9t_2)\equiv 0\mod 3^3 \Leftrightarrow 47 + t_2\equiv 0\mod 3 \Leftrightarrow 2 + t_2\equiv 0\mod 3 \Leftrightarrow t_2 = 1\mod 3.
		\end{align*}
		Logo, $a_2 = 4 + 9\cdot 1 = 13$. Note que
		\begin{align*}
		f(13) &= 30267 = 1121\cdot 27\equiv 0\mod 27 \\
		f'(13) &= 9049 \equiv 1\not\equiv 0\mod 3
		\end{align*}
		e portanto podemos aplicar o lema de Hensel mais uma vez. Então, queremos $t_3$ tal que
		\begin{align*}
		f(13+27t_3)\equiv 0\mod 3^4\Leftrightarrow 1121 + 1\cdot t_3\equiv 0\mod 3 \Leftrightarrow t_3 = 1\mod 3.
		\end{align*}
		Logo, $a_3 = 13 + 27\cdot 1 = 40$ é raiz de $f(x) = x^4 + 10x^2 + x + 3$ módulo $3^4$.  
	\end{proof}
	
	
	
	
	
	
	
	
	
	
\end{document}